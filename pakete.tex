
\usepackage[ngerman]{babel}

% Kodierung fuer unixoide Systeme und Windowssysteme.

\usepackage[utf8]{inputenc}

% Schriftart Times New Roman.

%\usepackage{times}

% Erweitert den Zeichenvorrat, so dass z.B. auch Umlaute im PDF-Dokument
% gefunden werden.

\usepackage[T1]{fontenc}

% Zur Einbindung von Bildern.

\usepackage{graphicx}

% Erweiterte enumerate-Umgebung.

\usepackage{enumerate}

% Verschiedene Pakete, die nuetzlich sind um Mathematik in LaTeX zu setzen.

\usepackage{amsmath, amssymb, amsthm, dsfont}

% Indentation

\usepackage{scrextend}

% Spaces for newcommand

\usepackage{xspace}

\usepackage{mathtools}

% tikz
\usepackage{tikz}
\usepackage{tikz-cd}
\usetikzlibrary{quotes,babel,angles}
%\usetikzlibrary{arrows.meta}% arrow tip library
%\usetikzlibrary{bending}% better arrow head for bended lines

% Hyphenation of wordas that already contain a hyphen
% http://tex.stackexchange.com/questions/2706/adequate-hyphenation-of-words-already-containing-a-hyphen
% Example:
% \textsc{Alexandroff}\hyp{}\textsc{Urysohn}\hyp{}Kompaktheit

\usepackage{hyphenat}

% Hyperlinks

%\usepackage{hyperref}
\usepackage[pdfauthor={Fabian Gabel},%
pdftitle={Seminar Lie-Algebren - Wurzelsysteme: einfache Wurzeln, die Weyl-Gruppe und Irreduzibilität},%
%pagebackref=true,%
%pdftex
]{hyperref}

%%%%%%%%%%%%%%%%
% Seitenlayout %
%%%%%%%%%%%%%%%%

% DIV# gibt den Divisor für die Layoutberechnung an.
% Vergrößern des Divisors vergrößert den Textbereich.
% BCOR#cm gibt die Breite des Bundstegs an.
\usepackage[DIV14,BCOR2cm]{typearea}

% Abstand obere Blattkante zur Kopfzeile ist 2.54cm - 15mm
\setlength{\topmargin}{-15mm}

% Keine Einrueckung nach einem Absatz.

\parindent 0pt

% Abstand zwischen zwei Abs\"atzen.

%\parskip 12pt

% Zeilenabstand.

\linespread{1.25}

% Inhaltsverzeichnis erstellen.

\usepackage{makeidx}
\makeindex
