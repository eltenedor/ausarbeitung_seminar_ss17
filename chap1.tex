\section{Kategorientheoretische Grundlagen (Fortsetzung)}

In diesem Abschnitt füllen wir das Vokabelheft mathematischer Definitionen mit weiteren Begriffen aus der Kategorientheorie.

\subsection{Funktoren, universelle Morphismen und Morphismen von Funktoren}

\begin{defn}[Covarianter Funktor]
  
\end{defn}

Contravarianter Funktor

\begin{ex}
  \begin{itemize}
    \item Konstanter Funktor
    \item Vergissfunktor
    \item Dualisierender Funktor
    \item Inklusionsfunktor
    \item Identitätsfunktor $\I_\C$.
  \end{itemize}
\end{ex}

\begin{defn}[Universelle Abbildung]
  Seien $\A$ und $\B$ Kategorien, $\F \colon \A \to \B$ ein Funktor und $B \in |\B|$.
  Ein Paar $(u, A)$ mit $A \in |\A|$ und $u \colon B \to \F(A)$ heißt \emph{universelle Abbildung für $B$ bezüglich $\F$}, falls für alle $A' \in |\A|$ und alle $f \colon B \to \F(A')$ genau ein $\A$-Morphismus $\overline f \colon A \to A'$ existiert so dass das Diagramm
  $$
  \begin{tikzcd}[row sep=2.5em]
    B \arrow[rd,"u"] \arrow[rr, "f"] &&\F(A') \\
    &\F(A) \arrow[ru, "\F(\overline{f})"]
  \end{tikzcd}
  $$
  kommutiert.
  Entsprechend bezeichnet man ein Paar $(A,u)$ mit $A \in |\A|$ und $u \colon \F(A) \to B$ als \emph{co-universelle Abbildung für $B$ bezüglich $\F$}, falls $(u^*, A)$ eine universelle Abbildung für $B$ bezüglich des Funktors $\F^* \colon \A^* \to \B^*$ ist.
  Dies bedeutet, dass für alls $A' \in |\A|$ und jeden $\B$-Morphismus $f \colon \F^(A') \to B$ ein eindeutiger $\A$-Morphismus existiert, so dass das Diagramm
  $$
  \begin{tikzcd}[row sep=2.5em]
    B  && \ \arrow[ll, "f"] \F(A') \\
    & \arrow[lu,"u"] \F(A)\arrow[ru, "\F(\overline{f})"] & 
  \end{tikzcd}
  $$
  kommutiert.
\end{defn}

Im folgenden Lemma beschreiben wir das Verhalten (co-)universeller Abbildungen unter Verknüpfung mit Isomorphismen.

\begin{lem}
  \label{lem:universalCircIso}
  Seien $\A$ und $\B$ Kategorien, $\F \colon \A \to \B$ ein Funktor und $B \in |\B|$ und $(u,A)$ eine universelle Abbildung für $B$ bezüglich $\F$.
  Sei nun $v \colon \A \to \underline A$ ein $\A$-Isomorphismus, dann ist auch $(\F(v) \circ u, \underline A )$ eine universelle Abbildung für $B$ bezüglich $\F$.

  Ist $(A, u)$ eine couniverselle Abbildung für $B$ bezüglich $\F$, so ist auch $(\underline A, u \circ \F(v^{-1}))$ eine couniverselle Abbildung für $B$ bezüglich $\F$.
\end{lem}

\begin{proof}
  Sei $f \colon B \to \F(A')$ ein $\B$-Morphismus. 
  So existiert aufgrund der Eigenschaften von $u$ genau ein $\A$-Morphismus $\overline f \colon A \to A'$ mit $f = \F(\overline f) \circ u$.
  Aufgrund der Eindeutigkeit von $f$ existiert somit genau ein $g \coloneqq v^{-1} \circ \overline f \colon \underline A \to A'$, sodass das Diagramm   
  $$
  \begin{tikzcd}[row sep=2.5em]
    B \arrow[rd,"u"] \arrow[rr, "f"] &&\F(A') \\
    &\F(A) \arrow[ru, "\F(\overline{f})"] \arrow[rd,"\F(v)"] \\
    &&\F(\underline A) \arrow[uu,"\F(g)"]
  \end{tikzcd}
  $$
  mitsamt seiner Unterdiagramme kommutiert.
  Über ein analoges Argument zeigt man, dass im Falle einer couniversellen Abbildung das Diagramm
  $$
  \begin{tikzcd}[row sep=2.5em]
    B  & & \F(A') \arrow[ll, "f"]  \\
    & \F(A)\arrow[ru, "\F(\overline{f})"] \arrow[lu,"u"] \arrow[rd, "\F(v)"]  & \\
    & & \F(\underline A) \arrow[uu, "\F(g)"]
  \end{tikzcd}
  $$
  kommutiert.
\end{proof}

wir zeigen nun gewissermaßen die Umkehrung des vorangehenden Lemmas, nämlich, dass universelle Abbildungen bereits eindeutig bis auf Isomorphie sind.

\begin{prop}[\cite{preuss}, 2.1.6]
  Seien $(u,A)$ und $(u',A')$ universelle Abbildungen für $B \in |\B|$ bezüglich $\F \colon \A \to \B$.
  Dann existiert ein Isomorphismus $f \colon A \to \A'$, sodass das Diagramm
  $$
  \begin{tikzcd}[row sep=2.5em]
    B \arrow[rd,"u'"] \arrow[rr, "u"] &&\F(A) \\
    &\F(A') \arrow[ru, "\F(\overline{f})"]
  \end{tikzcd}
  $$
  kommutiert.
\end{prop}

\begin{ex}
  \begin{itemize}
    \item T0-ifizierung
    \item Stone-Cech Kompaktifizierung
    \item Vergissfunktor
  \end{itemize}
\end{ex}

\begin{defn}[Natürliche Transformationen/Morphismen von Funktoren]
  
\end{defn}

\subsection{Adjungierte Funktoren}

Morphismen zwischen Identitätsfunktor und einer Verkettung von Funktoren

\begin{thm}[\cite{preuss}, 2.1.12]
  Sei $\F \colon \A \to B$ ein Funktor mit der Eigenschaft, dass für alle $B \in |\B|$ eine universelle Abbildung $(u_B, A_B)$ bezüglich $F$ existiert.
  Dann existiert genau ein Funktor $\G \colon B \to \A$, sodass Folgendes gilt:
  \begin{enumerate}[(1)]
    \item $\G(B) = A_B$ für alle $B \in |\B|$.
    \item $u = (u_B) \colon \I_B \to \F \circ \G$ ist eine natürliche Transformation.
  \end{enumerate}
\end{thm}

\begin{kor}[\cite{preuss}, 2.1.12]
  Es existiert genau eine natürliche Transformationa $v = (v_A) \colon \G \circ F \to \I_A$, sodass das Folgende gilt:
  \begin{enumerate}[(a)]
    \item $\F(v_A) \colon u_{\F(A)} = \1_{\F(A)}$ für alle $A \in |\A|$.
    \item $v_{\G(B)} \circ \G(u_B) = \1_{\G(B)}$ für alle $B \in |\B|$.
  \end{enumerate}
\end{kor}

\begin{defn}[Linksadjungierter Funktor]
  Sind $\F \colon \A \to \B$ und $\G \colon \B \to \A$ Funktoren und $u = (u_B) \colon \I_\B \to \F \colon \G$ sowie $v = (v_A) \colon \G \circ \F \to \I_\A$ natürliche Transformationen mit den Eigenschaften
  \begin{enumerate}[(1)]
    \item $\F(v_A) \circ u_{\F(A)} = \1_{\F(A)}$ für alle $A \in |\A|$ und
    \item $v_{\G(B)} \circ \G(u_B) = \1_{\G(B)}$ für alle $B \in |\B|$,
  \end{enumerate}
  so nennen wir $\G$ den \emph{zu $\F$ linksadjungierten Funktor} und analog nennen wir $\F$ den \emph{zu $\G$ rechtsadjungierten Funktor}.
  Das Paar $(\G, \F)$ nennen wir ein \emph{Paar adjungierter Funktoren}.
\end{defn}

\begin{thm}[\cite{preuss}, 2.1.15]
  Ist $\G \colon \B \to \A$ ein zu $\F \colon \A \to \B$ linksadjungierter Funktor und $u = (u_B) \colon \I_\B \to \F \circ \G$ eine zugehörige natürliche Transformation, dann ist für alle $B \in |B|$ das Paar $(u_B, \G(B))$ eine universelle Abbildung bezüglich $\F$.
\end{thm}

\begin{bem}[Adjungierte Situation]
    
\end{bem}

\begin{ex}
  \begin{itemize}
    \item T0-ifizierung
    \item Stone-Cech
    \item Vergissfunktor
  \end{itemize}
\end{ex}


