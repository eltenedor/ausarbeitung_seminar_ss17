% Vorlage fuer Abschlussarbeiten am Fachbereich Mathematik der TU Darmstadt.
% Geeignet fuer Bachelorarbeiten, Masterarbeiten und Diplomarbeiten.

%\documentclass[12pt,a4paper,twoside]{report}
\documentclass[12pt,a4paper,twoside]{article}
% Benutzt man die Option draft, so kann man die Umbrueche ueberpruefen.

% Hier werden alle benoetigten Pakete und Einstellungen geladen. Auch
% hier sind Sie frei diese direkt in der Praeambel zu laden.

\usepackage[ngerman]{babel}
\usepackage[utf8]{inputenc}
\usepackage[T1]{fontenc}

% Zur Einbindung von Bildern.

%\usepackage{graphicx}

% Erweiterte enumerate-Umgebung.

\usepackage{enumerate}
\usepackage{amsmath, amssymb, amsthm, dsfont}
\usepackage{mathtools}
\usepackage{tikz}
\usepackage{tikz-cd}
\usetikzlibrary{quotes,babel,angles}
%\usepackage{hyperref}
\usepackage[pdfauthor={Fabian Gabel},%
pdftitle={Topologie-Seminar - Reflektionen \& Coreflektionen},%
%pagebackref=true,%
%pdftex
]{hyperref}

% Indentation

%\usepackage{scrextend}

% Spaces for newcommand

%\usepackage{xspace}


% tikz
%\usetikzlibrary{arrows.meta}% arrow tip library
%\usetikzlibrary{bending}% better arrow head for bended lines

% Hyphenation of wordas that already contain a hyphen
% http://tex.stackexchange.com/questions/2706/adequate-hyphenation-of-words-already-containing-a-hyphen
% Example:
% \textsc{Alexandroff}\hyp{}\textsc{Urysohn}\hyp{}Kompaktheit

% Hyperlinks


%%%%%%%%%%%%%%%%
% Seitenlayout %
%%%%%%%%%%%%%%%%

% DIV# gibt den Divisor für die Layoutberechnung an.
% Vergrößern des Divisors vergrößert den Textbereich.
% BCOR#cm gibt die Breite des Bundstegs an.
\usepackage[DIV14,BCOR2cm]{typearea}

% Abstand obere Blattkante zur Kopfzeile ist 2.54cm - 15mm
\setlength{\topmargin}{-15mm}

% Keine Einrueckung nach einem Absatz.

\parindent 0pt

% Abstand zwischen zwei Abs\"atzen.

%\parskip 12pt

% Zeilenabstand.

\linespread{1.25}

% Inhaltsverzeichnis erstellen.

\usepackage{makeidx}
\makeindex


% Hier werden Makros und Umgebungen eingebunden. Diese werden separat
% in der Datei befehle.tex definiert. Sie sind frei diese Befehle auch
% direkt in der Praeambel zu definieren.
% Makros
\numberwithin{equation}{section}

%Zahlkörper
\newcommand{\N}{\mathbb{N}} 
\newcommand{\Z}{\mathbb{Z}} 
\newcommand{\Q}{\mathbb{Q}} 
\newcommand{\R}{\mathcal{R}}
\newcommand{\C}{\mathcal{C}}
\newcommand{\K}{\mathbb{K}} 

%Funktionenräume
\newcommand{\CC}{\mathrm{C}} 
\newcommand{\Ell}{\mathrm{L}} 
\newcommand{\Li}{\mathfrak{L}}
\newcommand{\BB}{\mathrm{B}}
\newcommand{\DD}{\mathrm{D}}

%Mengen
\newcommand{\E}{\mathcal{E}}
\newcommand{\J}{\mathcal{J}}
\newcommand{\Sc}{\mathcal{S}}
\newcommand{\F}{\mathcal{F}}
\newcommand{\FF}{\mathrm{F}}
\newcommand{\FU}{\mathrm{F}_0}
\newcommand{\A}{\mathcal{A}}
\newcommand{\U}{\mathcal{U}}
\newcommand{\B}{\mathcal{B}}
\newcommand{\G}{\mathcal{G}}
\newcommand{\I}{\mathcal{I}}
\newcommand{\Pot}{\mathcal{P}}

%Kategorien
\newcommand{\FilDSUConv}{\mathbf{Fil\textbf{-}D\textbf{-}SUConv}}
\newcommand{\Sy}{\mathrm{S}}
\newcommand{\Unif}{\mathbf{Unif}}
\newcommand{\CFil}{\mathbf{CFil}}
\newcommand{\Fil}{\mathbf{Fil}}
\newcommand{\Conv}{\mathbf{Conv}}
\newcommand{\GConv}{\mathbf{GConv}}
\newcommand{\KConv}{\mathbf{KConv}}
\newcommand{\Lim}{\mathbf{KConv}}
\newcommand{\PrTop}{\mathbf{PrTop}}
\newcommand{\PrULim}{\mathbf{PrULim}}
\newcommand{\PsTop}{\mathbf{PsTop}}
\newcommand{\SUConv}{\mathbf{SUConv}}
\newcommand{\SULim}{\mathbf{SULim}}
\newcommand{\TPrTop}{\mathbf{TPrTop}}
\newcommand{\Top}{\mathbf{Top}}
\newcommand{\ULim}{\mathbf{ULim}}
\newcommand{\Mor}{\mathrm{Mor}}

%Sonstige
\newcommand{\co}{\mathrm{co}} % compact open
\newcommand{\ee}{\mathrm{e}} % compact open
\newcommand{\W}{\mathcal{W}} % Weyl-Gruppe
\newcommand{\T}{\mathbb{T}}
\newcommand{\St}{\mathfrak{S}}
\newcommand{\1}{\mathbf{1}}


\def\esssup{\qopname\relax m{\mathrm{ess}\,sup}}


% Umgebungen für Definitionen, Sätze, usw.

%\newtheorem{defn}{Definition}[section]
%\newtheorem{ex}{Beispiel}[chapter]

\theoremstyle{plain}
\newtheorem{thm}{Satz}[section]
\newtheorem{lem}[thm]{Lemma}
\newtheorem{kor}[thm]{Korollar}
\newtheorem{prop}[thm]{Proposition}
\newtheorem{ntion}{Notation}

\theoremstyle{definition}
\newtheorem{defn}[thm]{Definition}
\newtheorem{ex}[thm]{Beispiel}

\theoremstyle{remark}
\newtheorem*{bem}{Bemerkung}

\def\Satzrefname{Satz}

\DeclareMathOperator{\supp}{supp}
\DeclareMathOperator{\loc}{loc}
\DeclareMathOperator{\pv}{pv}
\DeclareMathOperator{\konv}{konv}
\DeclareMathOperator{\dist}{dist}
\DeclareMathOperator{\diag}{diag}
\DeclareMathOperator{\grad}{grad}
\DeclareMathOperator{\Spur}{Spur}
\DeclareMathOperator{\BUC}{BUC}
\DeclareMathOperator{\id}{id}
\DeclareMathOperator{\im}{im}
\DeclareMathOperator{\abs}{abs}
\DeclareMathOperator{\Op}{Op}
\DeclareMathOperator{\Aff}{Aff}
\DeclareMathOperator{\codim}{codim}
\DeclareMathOperator{\cstar}{C*}
\DeclareMathOperator{\stoplim}{stop-lim}
\DeclareMathOperator{\swoplim}{swop-lim}

\let\divsymb=\div % rename builtin command \div to \divsymb
\renewcommand{\div}[1]{\mathrm{div\,} #1} % for divergence
\renewcommand{\d}[1]{\ensuremath\, {\operatorname{d}\!{#1}}}
\newcommand{\norm}[1]{\lVert #1 \rVert}

\def\checkmark{\tikz\fill[scale=0.4](0,.35) -- (.25,0) -- (1,.7) -- (.25,.15) -- cycle;} 

\def\Xint#1{\mathchoice
{\XXint\displaystyle\textstyle{#1}}%
{\XXint\textstyle\scriptstyle{#1}}%
{\XXint\scriptstyle\scriptscriptstyle{#1}}%
{\XXint\scriptscriptstyle\scriptscriptstyle{#1}}%
\!\int}
\def\XXint#1#2#3{{\setbox0=\hbox{$#1{#2#3}{\int}$ }
\vcenter{\hbox{$#2#3$ }}\kern-.6\wd0}}
\def\ddashint{\Xint=}
\def\dashint{\Xint-}


\begin{document}
% Auf der Titelseite und im Inhaltsverzeichnis sollen keine
% Seitenzahlen dargestellt werden.
\pagestyle{empty}

% Hier wird die Titelseite eingebunden.
\begin{titlepage}
  \begin{center}
    \vspace{1cm}
    \includegraphics[width=0.5\linewidth]{TU_Darmstadt_Logo.pdf}
    \vspace{12pt}
    
    \large{Fachbereich Mathematik}
    \vspace{2cm}
    
    \large{Topologie-Seminar im Sommersemester 2017}
    \vspace{2cm}

    \huge{Reflektionen \& Coreflektionen}
    
    \vspace*{2cm}    
    
		\large
                \href{mailto:gabel@mathematik.tu-darmstadt.de}{Fabian Gabel}
    \vspace*{.5cm}

    01.06.2017 \\
    \vspace*{1cm}

    Veranstalter: Dr. rer. nat. Ren\'e Bartsch

    \vspace*{.5cm}

    %Zweiter Gutachter: Name des zweiten Gutachters\\[2ex]
    \tiny{Version vom \today}
  \end{center}
\end{titlepage}
\vspace*{\fill}


% Inhaltsverzeichnis erstellen.
\tableofcontents

% Ab sofort werden Seitenzahlen in der Kopfzeile angezeigt.
\pagestyle{headings}

\section{Kategorientheoretische Grundlagen (Fortsetzung)}

In diesem Abschnitt füllen wir das Vokabelheft mathematischer Definitionen mit weiteren Begriffen aus der Kategorientheorie.

\subsection{Funktoren, universelle Morphismen und natürliche Transformationen}

Wie schon in der Einleitung angekündigt, werden Funktoren in diesem Teil des Seminars eine prominente Rolle spielen, denn wir wollen unterschiedliche Kategorien zueinander in Relation setzen und Abbildungen die dies auf \emph{natürliche} Weise schaffen, nennen wir Funktoren.

\begin{defn}[Covarianter Funktor]
  Seien $\C$ und $\De$ Kategorien und $\F_1 \colon |\C| \to |\De|$ and $\F_2 \colon \Mor_\C\to \Mor_\De$. 
  Dann nennen wir das Quadrupel $\F = (\C, \De, \F_1, \F_2)$ einen (\emph{covarianten}) \emph{Funktor} von $\C$ nach $\De$, falls folgende Bedingungen erfüllt sind
  \begin{enumerate}[F1)]
    \item $f \in [A, B]_\C$ impliziert $\F(f) \in [\F(A), \F(B)]_\De$.
    \item $\F(f \circ g) = \F(f) \circ \F(g)$, falls $f \circ g$ definiert ist.
    \item $\F(1_A) = 1_{\F(A)}$ für alle $A \in |\C|$.
  \end{enumerate}
  Abkürzend schreiben wir im Folgenden auch $\F \colon \C \to \De$.
\end{defn}

Aufgrung der Eigenschaften F2) und F3) bezeichnet man Funktoren machmal auch als Homomorphismen von Morphismen.

Ein \emph{contravarianter Funktor} $\F \colon \C \to \De$ ist gerate ein covarianter Funktor von $\C^* \to \De$, was gerade bedeutet, dass die zusätzlich zu F1) folgende modifizierten Bedingungen gelten:
\begin{enumerate}[F1')]
  \item $f \in [A, B]_\C$ impliziert $\F(f) \in [\F(B), \F(A)]_\De$.
  \item $\F(f \circ g) = \F(g) \circ \F(f)$, falls $f \circ g$ existiert.
\end{enumerate}

Wir beleben den Begriff des Funktors nun, indem wir bekannte Sachverhalte kategorientheoretisch unter die Lupe nehmen.

\begin{ex}
\begin{enumerate}[a)]
    \item Konstanter Funktor: Sind $\C$ und $\De$ Kategorien und $X \in |\De|$.
      So lässt sich für alle $A \in |\C|$ und alle $f \in \Mor_\C$ durch $\F(A) = X$ und $\F(f) = 1_X$ ein Funktor definieren, welcher sowohl covariant als auch contravariant ist.
    \item Vergissfunktor: Ist $\C$ ein (topologisches) Konstrukt, so lässt sich ein Funktor $\F \to \Set$ definieren durch $\F((X, \xi)) = X$ und $\F(f) = f$.
    \item Dualisierender Funktor $\Delta_\C$: Es lässt sich ein Funktor $\F \colon \C \to \C^*$ definieren durch $\F(X) = X$ und $F(f) = f^*$. Dieser ist natürlich contravariant.
    \item Dualer Funktor: Ist $\F \colon \C \to \De$ ein Funktor, so erhält man den zugehörigen dualen Funtor als $\Delta_{\De} \circ F \circ \Delta_{\C^*}$
    \item Inklusionsfunktor: Sei $\C$ eine Kategorie und $\A$ eine \emph{Unterkategorie}, also eine Kategorie für die gilt
      \begin{enumerate}
        \item $|\A| \subset |\C|$,
        \item $[A, B]_\A \subset [A,B]_\C$ für alle $(A, B) \in |\A| \times |\A|$,
        \item Die Komposition von Morphismen in $\A$ stimmt mit der Komposition in $\C$ überein und die der Identitätsmorphismus ist auch derselbe.
      \end{enumerate}
      Git zu alledem sogar $[A, B]_\A = [A, B]_\C$, so bezeichnen wir $\A$ als \emph{volle} Unterkategorie.
    \item Identitätsfunktor $\I_\C$: Dieser Funktor $\F \colon \C \to \C$ wird durch $\F(X) = X$ und $\F(f) = f$ definiert.
  \end{enumerate}
\end{ex}

\begin{defn}[Universelle Abbildung]
  Seien $\A$ und $\B$ Kategorien, $\F \colon \A \to \B$ ein Funktor und $B \in |\B|$.
  Ein Paar $(u, A)$ mit $A \in |\A|$ und $u \colon B \to \F(A)$ heißt \emph{universelle Abbildung für $B$ bezüglich $\F$}, falls für alle $A' \in |\A|$ und alle $f \colon B \to \F(A')$ genau ein $\A$-Morphismus $\overline f \colon A \to A'$ existiert so dass das Diagramm
  $$
  \begin{tikzcd}[row sep=2.5em]
    B \arrow[rd,"u"] \arrow[rr, "f"] &&\F(A') \\
    &\F(A) \arrow[ru, "\F(\overline{f})"]
  \end{tikzcd}
  $$
  kommutiert.
  Entsprechend bezeichnet man ein Paar $(A,u)$ mit $A \in |\A|$ und $u \colon \F(A) \to B$ als \emph{co-universelle Abbildung für $B$ bezüglich $\F$}, falls $(u^*, A)$ eine universelle Abbildung für $B$ bezüglich des Funktors $\F^* \colon \A^* \to \B^*$ ist.
  Dies bedeutet, dass für alls $A' \in |\A|$ und jeden $\B$-Morphismus $f \colon \F^(A') \to B$ ein eindeutiger $\A$-Morphismus existiert, so dass das Diagramm
  $$
  \begin{tikzcd}[row sep=2.5em]
    B  &&  \arrow[ll, "f"] \F(A')\arrow[ld, "\F(\overline{f})"]  \\
    & \arrow[lu,"u"]\F(A) & 
  \end{tikzcd}
  $$
  kommutiert.
\end{defn}

Im folgenden Lemma beschreiben wir das Verhalten (co-)universeller Abbildungen unter Verknüpfung mit Isomorphismen.

\begin{lem}
  \label{lem:universalCircIso}
  Seien $\A$ und $\B$ Kategorien, $\F \colon \A \to \B$ ein Funktor und $B \in |\B|$ und $(u,A)$ eine universelle Abbildung für $B$ bezüglich $\F$.
  Sei nun $v \colon A \to \underline A$ ein $\A$-Isomorphismus, dann ist auch $(\F(v) \circ u, \underline A )$ eine universelle Abbildung für $B$ bezüglich $\F$.

  Ist $(A, u)$ eine couniverselle Abbildung für $B$ bezüglich $\F$, so ist auch $(\underline A, u \circ \F(v^{-1}))$ eine couniverselle Abbildung für $B$ bezüglich $\F$.
\end{lem}

\begin{proof}
  Sei $f \colon B \to \F(A')$ ein $\B$-Morphismus. 
  So existiert aufgrund der Eigenschaften von $u$ genau ein $\A$-Morphismus $\overline f \colon A \to A'$ mit $f = \F(\overline f) \circ u$.
  Aufgrund der Eindeutigkeit von $f$ existiert somit genau ein $g \coloneqq v^{-1} \circ \overline f \colon \underline A \to A'$, sodass das Diagramm   
  $$
  \begin{tikzcd}[row sep=2.5em]
    B \arrow[rd,"u"] \arrow[rr, "f"] &&\F(A') \\
    &\F(A) \arrow[ru, "\F(\overline{f})"] \arrow[rd,"\F(v)"] \\
    &&\F(\underline A) \arrow[uu,"\F(g)"]
  \end{tikzcd}
  $$
  mitsamt seiner Unterdiagramme kommutiert.
  Über ein analoges Argument zeigt man, dass im Falle einer couniversellen Abbildung das Diagramm
  $$
  \begin{tikzcd}[row sep=2.5em]
    B  & & \F(A') \arrow[ll, "f"]  \\
    & \F(A)\arrow[ru, "\F(\overline{f})"] \arrow[lu,"u"] \arrow[rd, "\F(v)"]  & \\
    & & \F(\underline A) \arrow[uu, "\F(g)"]
  \end{tikzcd}
  $$
  kommutiert.
\end{proof}

wir zeigen nun gewissermaßen die Umkehrung des vorangehenden Lemmas, nämlich, dass universelle Abbildungen bereits eindeutig bis auf Isomorphie sind.

\begin{prop}[\cite{preuss}, 2.1.6]
  Seien $(u,A)$ und $(u',A')$ universelle Abbildungen für $B \in |\B|$ bezüglich $\F \colon \A \to \B$.
  Dann existiert ein Isomorphismus $f \colon A \to \A'$, sodass das Diagramm
  $$
  \begin{tikzcd}[row sep=2.5em]
    B \arrow[rd,"u'"] \arrow[rr, "u"] &&\F(A) \\
    &\F(A') \arrow[ru, "\F(\overline{f})"]
  \end{tikzcd}
  $$
  kommutiert.
\end{prop}

Widmen wir unsere Aufmerksamkeit nun ein paar famosen Beispielen, um die eigeführten Prinzipien bei der Arbeit zu bestaunen.

\begin{ex}
  \begin{enumerate}
    \item T0-ifizierung: 
    \item Stone-Cech Kompaktifizierung: Sei $\Tych$ die Kategorie der Tychonoff-Räume, und $\CompHaus$ die Kategorie der kompakten Hausdorff-Räume.
      Ferner bezeichnen wir für $X \in |\Tych|$ mittels $\beta(X)$ seine Stone-\v{C}ech-Kompatifizierung und mit $e_X \colon X \to \beta(X)$ die entsprechende Einbettung. Dann ist nach dem Satz von Stone-\v{C}ech \cite[5.4.8]{bartsch} das Paar $(e_X, \beta(X))$ eine universelle Abbildung bezüglich des Inklusionsfunktors $\F_e \colon \CompHaus \to \Tych$.
      Ein entsprechend angepasstes kommutatives Diagramm liefert Gewissheit. Hierbei sei $Y \in \CompHaus$ und $f \in [X, \F_e(Y)]_{\Tych}$:
      $$
      \begin{tikzcd}[row sep=2.5em]
        X \arrow[rd,"e_X"] \arrow[rr, "f"] &&\F_e(Y) = Y \\
        &\F_e(\beta(X)) = \beta(X) \arrow[ru, "\F_e(\overline{f})"]
      \end{tikzcd}
      $$
    \item Vergissfunktor
  \end{enumerate}
\end{ex}

\begin{defn}[Natürliche Transformationen/Morphismen von Funktoren]
  
\end{defn}

\subsection{Adjungierte Funktoren}

Morphismen zwischen Identitätsfunktor und einer Verkettung von Funktoren

\begin{thm}[\cite{preuss}, 2.1.12]
  Sei $\F \colon \A \to B$ ein Funktor mit der Eigenschaft, dass für alle $B \in |\B|$ eine universelle Abbildung $(u_B, A_B)$ bezüglich $F$ existiert.
  Dann existiert genau ein Funktor $\G \colon B \to \A$, sodass Folgendes gilt:
  \begin{enumerate}[(1)]
    \item $\G(B) = A_B$ für alle $B \in |\B|$.
    \item $u = (u_B) \colon \I_B \to \F \circ \G$ ist eine natürliche Transformation.
  \end{enumerate}
\end{thm}

\begin{kor}[\cite{preuss}, 2.1.12]
  Es existiert genau eine natürliche Transformationa $v = (v_A) \colon \G \circ F \to \I_A$, sodass das Folgende gilt:
  \begin{enumerate}[(a)]
    \item $\F(v_A) \circ u_{\F(A)} = \1_{\F(A)}$ für alle $A \in |\A|$.
    \item $v_{\G(B)} \circ \G(u_B) = \1_{\G(B)}$ für alle $B \in |\B|$.
  \end{enumerate}
\end{kor}

\begin{defn}[Linksadjungierter Funktor]
  Sind $\F \colon \A \to \B$ und $\G \colon \B \to \A$ Funktoren und $u = (u_B) \colon \I_\B \to \F \colon \G$ sowie $v = (v_A) \colon \G \circ \F \to \I_\A$ natürliche Transformationen mit den Eigenschaften
  \begin{enumerate}[(1)]
    \item $\F(v_A) \circ u_{\F(A)} = \1_{\F(A)}$ für alle $A \in |\A|$ und
    \item $v_{\G(B)} \circ \G(u_B) = \1_{\G(B)}$ für alle $B \in |\B|$,
  \end{enumerate}
  so nennen wir $\G$ den \emph{zu $\F$ linksadjungierten Funktor} und analog nennen wir $\F$ den \emph{zu $\G$ rechtsadjungierten Funktor}.
  Das Paar $(\G, \F)$ nennen wir ein \emph{Paar adjungierter Funktoren}.
\end{defn}

\begin{thm}[\cite{preuss}, 2.1.15]
  Ist $\G \colon \B \to \A$ ein zu $\F \colon \A \to \B$ linksadjungierter Funktor und $u = (u_B) \colon \I_\B \to \F \circ \G$ eine zugehörige natürliche Transformation, dann ist für alle $B \in |B|$ das Paar $(u_B, \G(B))$ eine universelle Abbildung bezüglich $\F$.
\end{thm}

\begin{bem}[Adjungierte Situation]
    
\end{bem}

\begin{ex}
  \begin{itemize}
    \item T0-ifizierung
    \item Stone-Cech
    \item Vergissfunktor
  \end{itemize}
\end{ex}



\section{Reflektive und coreflektive Unterkategorien}

In diesem Abschnitt wollen wir uns speziell mit Inklusionsfunktoren und ihren Adjungierten beschäftigen. 
In der freien Wildbahn treten Inklusionsfunktoren bei der Betrachtung von Unterkategorien auf.

\begin{defn}[Reflektive Unterkategorie]
  Sei $A$ eine Unterkategorie einer Kategorie $\C$ und $\F_e \colon \A \to \C$ der Inklusionsfunktor.
  Dann nennen wir $\A$ \emph{reflektiv} in $\C$ genau dann, wenn eine der folgenden äquivalenten Bedingungen erfüllt ist:
  \begin{enumerate}[(1)]
    \item $\F_e$ besitzt den Linksadjungierten Funktor $\R$.
    \item Jedes $X \in |C|$ besitzt eine universelle Abbildung $(r_X, X_\A)$ bezüglich $\F_e$.
  \end{enumerate}
  Den Funktor $\R$ nennen wir dann einen \emph{Reflektor}, die Morphismen $r_X \colon X \to X_\A$ nennen wir Reflektionen von $X$ bezüglich $\A$.

  Durch Dualisierung erhalten wir einen weiteren Begriff.
  Wir nennen $\A$ \emph{coreflektiv} in $\C$, genau dann, wenn $\A^*$ reflektiv ist in $\C^*$. 
\end{defn}

\begin{defn}
  In der Situation von Definition ? nennen wir $\A$ \emph{epireflektiv/ extremal epireflektiv/ bireflektiv in} $\C$, falls $\A$ reflektiv in $\C$ ist und der für alle $X \in |\C|$ existierende $\C$-morphismus $r_X \colon X \to X_\A$ ein Epimorphismus/ extremaler Epimorphismus / Bimorphismus ist.
  Die Morphismen $r_X$ nennen wir \emph{Epireflektionen/ extremale Epireflektionen/ Bireflektionen}.
\end{defn}

  \begin{itemize}
    \item Reflektive Unterkategorie
    \item Reflektor
    \item epireflektive, extremal epireflektiv, bireflektive Unterkategorie
    \item Reflektionen
  \end{itemize}

\begin{defn}
  Wir nennen ein Objekt $S$ eine Kategorie $\C$ \emph{Separator}, falls für alle paarweise verschiedenen Morphismen $f,g \colon A \to B$ mit gleichem Definitions und Wertebereich ein Morphismus $h \colon S \to A$ existiert mit der Eigenschaft, dass $f \circ h \neq g \circ h$.
\end{defn}

Wir halten fest, dass jedes Objekt $(X,\xi)$ eines topologischen Konstrukts $\C$ mit $X \neq \emptyset$ ein Separator ist.

\begin{bem}[2.2.11, S.65]
  Jedes coreflektive, volle und unter Isomorphie abgeschlossene Unterkonstrukt $\A$ eines topologischen Konstrukts $\C$ ist bicoreflektiv, falls $|\A|$ mindestens ein Element mit nicht leerer zugrunde liegender Menge enthält.

  In diesem Fall ist die zu $(X, \xi) \in \C$ gehörige Coreflektion $c_X \colon (Y_\A, \eta_\A) \to (X, \xi)$ bijektiv.
  Nach \cite[1.2.2.7]{preuss} existiert eine $\C$-Struktur $\xi_\A$ auf $X$, sodass $c_X \colon (Y_\A, \eta_\A) \to (X, \xi_\A)$ ein Isomorphismus ist.
  Da $\A$ nach Voraussetzung abgeschlossen unter Isomorphismen ist, gilt $(X, \xi_\A)\in \A$.
  Wir zeigen nun, dass $\xi_\A$ die gröbste aller $\C$-Strukturen $\xi'$ ist, für die einerseits $\xi' \leq \xi$ und andererseits $(X, \xi') \in \A$ gilt.

\end{bem}

\begin{thm}[2.2.12, S.66]
  
\end{thm}


\section{Konvergenzstrukturen und uniforme Konvergenzstrukturen}

In diesem letzten Abschnitt betrachten wir nun unterschiedliche Konvergenzstrukturen und uniforme Strukturen durch die kategorientheoretische Brille, mit dem Ziel diese untereinander in Beziehung zu setzen und die Verbindung von Konvergenzstrukturen und uniformen Strukturen zu verstehen.


\subsection{Konvergenzstrukturen}

Zunächst einmal halten wir fest, welche Konvergenzstrukturen für uns interessant sein werden.

\begin{defn}[GKonv und seine Kinder]
  Die Kategorie $\GConv$ der verallgemeinerten Konvergenzräume mit stetigen Abbildungen setzt sich wie folgt zusammen:
  \begin{enumerate}[a)]
    \item Für jede Menge $X$ sei $\FF(X)$ die Menge aller Filter auf $X$. 
      Ein \emph{verallgemeinerter Konvergenzraum} ist ein Paar $(X,q)$, wobei $X$ eine Menge und $q \subset \FF(X) \times X$ eine Relation von Filtern und Punkten (gegen die sie \emph{konvergieren}) ist. 
      Zusätzlich sollen folgende Eigenschaften erfüllt sein:
      \begin{enumerate}[C1)]
        \item $(\dot x, x) \in q$ für alle $x \in X$; \emph{alle Einpunktfilter konvergieren gegen ihren Erzeuger}.
        \item $(\G, x) \in q$, falls $(F,x) \in q$ und $G \supset F$; \emph{Oberfilter konvergenter Filter, erben Grenzwerte}
      \end{enumerate}
    \item Eine Abbildung $f \colon (X,q) \to (X',q')$ zwischen verallgemeinerten Konvergenzräumen heißt \emph{stetig}, falls für alle $(\F,x) \in q$ auch $(f(\F), f(x)) \in q'$ gilt.
  \end{enumerate}
  Ein verallgemeinerter Konvergenzraum heißt
  \begin{enumerate}[a)]
    \item[c)] \emph{Kent Konvergenzraum}, falls folgende Bedingung erfüllt ist:
      \begin{enumerate}
        \item[C3)] $(\F \cap \dot x, x) \in q$, falls $(F, x) \in q$; \emph{Abgeschlossenheit bezüglich endlicher Durchschnitte mit Einpunktfiltern}.
      \end{enumerate}
    \item[d)] \emph{Limesraum}, falls folgende Bedingung erfüllt ist:
      \begin{enumerate}
        \item[C4)] $(\F \cap \G, x)$, falls $(\F,x) \in q$ und $(\G,x) \in q$; \emph{Abgeschlossenheit bezüglich endlicher Durchschnitte}
      \end{enumerate}
    \item[e)] \emph{Pseudotopologischer Raum}, falls folgende Bedingung erfüllt ist:
      \begin{enumerate}
        \item[C5)] $(\F,x) \in q$, falls $(\U, x) \in q$ für alle Ultrafilter $\U \supset \F$.
      \end{enumerate}
    \item[f)] \emph{Prätopologischer Raum}, falls folgende Bedingung erfüllt ist:
      \begin{enumerate}
        \item[C6)] $(\U_q(x), x) \in q$ für alle $x \in X$, wobei $\U_q(x) \coloneqq \bigcap\{ \F \in F(x) \colon (\F, x) \in q\}$
      \end{enumerate}
  \end{enumerate}
  Ein prätopologischer Raum $(X,q)$ heißt 
  \begin{enumerate}[a)]
    \item[g)] \emph{topologischer Raum}, falls die folgende Bedingung erfüllt ist:
      \begin{enumerate}
        \item[C7)] Für alle $U \in \U_q(x)$ existiert ein $V \in \U_q(x)$ sodass $U \in \U_q(y)$ für alle $y \in V$ gilt.
      \end{enumerate}
  \end{enumerate}
\end{defn}

Die eben definierten Klassen definieren volle und unter Isomorphie abgeschlossene Unterkonstrukte von $\GConv$, welche wir im Folgenden mit $\Lim$, $\PsTop$, $\PrTop$ und $\TPrTop$ bezeichnen werden.

\begin{bem}[\cite{preuss}, 2.3.1.2]
Entsprechend der Definitionsreihenfolge existiert auch eine Inklusionskette der definierten Räumlichkeiten:
  $$
  \GConv \supset \KConv \supset \Lim \supset \PsTop \supset \PrTop \supset \TPrTop.
  $$
\end{bem}

\begin{proof}
  Jeder topologische Raum ist per definitionem ein prätopologischer Raum.

  Jeder prätopologische Raum ist ein pseutopologischer Raum: 
  Ist nämlich $(X,q) \in |\PrTop|$, so gilt $(\F,x) \in q$ genau dann, wenn $\F \supset U_q(x)$. Setzen wir nun voraus, dass $(\U,x) \in q$ für alle Ultrafilter $\U \supset \F$ gilt, so folgt aus
  $$
  \U_q(x) \subset \bigcap \{ \U \colon \U \in \FU(\F)\} = \F,
  $$
  wobei $\FU(\F)$ die Menge der Oberultrafiter von $\F$ bezeichne, die Behauptung durch Anwendung von C2).

  Jeder pseutotopologische Raum ist ein Limesraum: 
  Angenommen C4) sei nicht erfüllt für einen Limesraum $(X,q)$, so existieren Filter $\F,\G \in \FF(X)$ mit $(\F,x) \in q$ und $(\G,x) \in q$ aber $(\F \cap \G,x) \not\in q$.
  Folglich besitzt $(\F \cap \G, x)$ nach C5) ein Oberultrafilterfilter $(\U,x) \not\in q$.
  Insbesondere gilt nach C2) $\U \not\supset \F$ und $\U \not\supset \G$, es existieren also $F \in \F$ und $G \in \G$ mit $F,G \not \in \U$.
  Da $\U$ jedoch ein Oberfiter von $\F \cap \G$ ist, enthält er $F \cup G$ und aufgrund der Ultrafiltereigenschaft $F$ oder $G$ im Widerspruch zu $F, G \not\in \U$. 

  Jeder Limesraum ist ein Kent Konvergenzraum: Dies folgt sofort aus C1).

  Dass jeder Kent Konvergenzraum ein verallgemeinerter Konvergenzraum ist, ist wie bei allen anderen Konvergenzstrukturen Teil der Definition.
\end{proof}

\begin{prop}
  $\KConv$ ist bireflektives und bicoreflektives Unterkonstrukt von $\GConv$.
\end{prop}

\begin{prop}[\ref{preuss}, 2.3.1.5]
  Restliche Unterkonstrukte sind bireflektiv.
\end{prop}

\subsection{Uniforme Konvergenzstrukturen}

In 

\subsection{Das fehlende Puzzlestück}



\bibliographystyle{alpha}
\bibliography{seminar17_fabian_gabel}
      
\end{document}
