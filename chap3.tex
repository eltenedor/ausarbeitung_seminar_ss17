\section{Konvergenzstrukturen und uniforme Konvergenzstrukturen}

In diesem letzten Abschnitt betrachten wir nun unterschiedliche Konvergenzstrukturen und uniforme Strukturen durch die kategorientheoretische Brille, mit dem Ziel diese untereinander in Beziehung zu setzen und die Verbindung von Konvergenzstrukturen und uniformen Strukturen zu verstehen.

\subsection{Konvergenzstrukturen}

\begin{defn}
  \begin{itemize}
    \item Verallgemeinerter Konvergenzraum
    \item Kent Konvergenzraum
    \item Limesraum
    \item Pseudotopologischer Raum
    \item Prätopologischer Raum
  \end{itemize}
\end{defn}

\begin{bem}[2.3.1.2, S.68, Kette der Konvergenzstrukturen]
  
\end{bem}

\begin{prop}
  $\KConv$ ist bireflektives und bicoreflektives Unterkonstrukt von $\GConv$.
\end{prop}

\begin{prop}[2.3.1.5]
  Restliche Unterkonstrukte sind bireflektive.
\end{prop}

\subsection{Uniforme Konvergenzstrukturen}

\subsection{Das fehlende Puzzlestück}

