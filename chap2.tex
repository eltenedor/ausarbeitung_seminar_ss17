\section{Reflektive und coreflektive Unterkategorien}

In diesem Abschnitt wollen wir uns speziell mit Inklusionsfunktoren und ihren Adjungierten beschäftigen. 
In der freien Wildbahn treten Inklusionsfunktoren bei der Betrachtung von Unterkategorien auf.

\begin{defn}[Reflektive Unterkategorie]
  Sei $A$ eine Unterkategorie einer Kategorie $\C$ und $\F_e \colon \A \to \C$ der Inklusionsfunktor.
  Dann nennen wir $\A$ \emph{reflektiv} in $\C$ genau dann, wenn eine der folgenden äquivalenten Bedingungen erfüllt ist:
  \begin{enumerate}[(1)]
    \item $\F_e$ besitzt den Linksadjungierten Funktor $\R$.
    \item Jedes $X \in |C|$ besitzt eine universelle Abbildung $(r_X, X_\A)$ bezüglich $\F_e$.
  \end{enumerate}
  Den Funktor $\R$ nennen wir dann einen \emph{Reflektor}, die Morphismen $r_X \colon X \to X_\A$ nennen wir Reflektionen von $X$ bezüglich $\A$.

  Durch Dualisierung erhalten wir einen weiteren Begriff.
  Wir nennen $\A$ \emph{coreflektiv} in $\C$, genau dann, wenn $\A^*$ reflektiv ist in $\C^*$. 
\end{defn}

\begin{defn}
  In der Situation von Definition ? nennen wir $\A$ \emph{epireflektiv/ extremal epireflektiv/ bireflektiv in} $\C$, falls $\A$ reflektiv in $\C$ ist und der für alle $X \in |\C|$ existierende $\C$-morphismus $r_X \colon X \to X_\A$ ein Epimorphismus/ extremaler Epimorphismus / Bimorphismus ist.
  Die Morphismen $r_X$ nennen wir \emph{Epireflektionen/ extremale Epireflektionen/ Bireflektionen}.
\end{defn}

\begin{defn}
  Wir nennen ein Objekt $S$ einer Kategorie $\C$ \emph{Separator}, falls für alle paarweise verschiedenen Morphismen $f,g \colon A \to B$ mit gleichem Definitions und Wertebereich ein Morphismus $h \colon S \to A$ existiert mit der Eigenschaft, dass $f \circ h \neq g \circ h$.
\end{defn}

Wir halten fest, dass jedes Objekt $(X,\xi)$ eines topologischen Konstrukts $\C$ mit $X \neq \emptyset$ ein Separator ist. 
Denn für zwei paarweise verschiedene Morphismem $f,g \colon (Y, \eta) \to (Z, \theta)$ unterscheiden sich die zugrundeliegenden Mengenabbildungen $f$ und $g$ zumindest schonmal in einem Punkt $y \in Y$. Betrachten wir nun die konstante Abbildung $h \colon X \to Y, h(x) = y$, so ist diese aufgrund der Voraussetzung $X \neq \emptyset$ wohldefiniert und zudem ein Morphismus.
Damit folgt sofort die Behauptung.

Es stellt sich heraus, dass die bloße Existenz von Separatoren weitere Eigenschaften der Coreflektionen in folgender Weise freilegt.

\begin{thm}[\cite{preuss}, 2.2.9]
  Sei $S$ ein Separator einer Kategorie $\C$ und $\A$ eine koreflektive Unterkategorie von $\C$, die $S$ enthält.
  Dann ist $\A$ bereits epicoreflektiv.
\end{thm}

Wir wissen also, wann eine coreflektive Unterkategorie epicoreflektiv ist. 
Der folgende Satz geht nun einen Schritt weiter zu bicoreflektiven Unterkategorien.

\begin{thm}
  Sei $\A$ eine epicoreflektive Unterkategorie von $\C$. 
  Ist $\A$ zuätzlich eine \emph{volle} Unterkategorie, so ist $\A$ bereits bicoreflektiv.
\end{thm}

\begin{bem}[2.2.11, S.65]
  Jedes coreflektive, volle und unter Isomorphie abgeschlossene Unterkonstrukt $\A$ eines topologischen Konstrukts $\C$ ist bicoreflektiv, falls $|\A|$ mindestens ein Element mit nicht leerer zugrunde liegender Menge enthält.

  Wir wollen im Folgenden zeigen, dass in diesem Fall die Coreflektionen eine sehr einfache Gestalt annehmen.  
  Für $(X, \xi) \in |\C|$ ist die entsprechende Coreflektion $c_X \colon (Y_\A, \eta_\A) \to (X, \xi)$ bijektiv.
  Nach \cite[1.2.2.7]{preuss} existiert eine $\C$-Struktur $\xi_\A$ auf $X$, sodass $c_X \colon (Y_\A, \eta_\A) \to (X, \xi_\A)$ ein Isomorphismus ist.
  Da $\A$ nach Voraussetzung abgeschlossen unter Isomorphismen ist, gilt $(X, \xi_\A)\in |\A|$.
  Zudem ist $\xi_\A$ die gröbste aller $\C$-Strukturen $\xi'$, für die einerseits $\xi' \leq \xi$ und andererseits $(X, \xi') \in \A$ gilt.

  Nach Lemma \ref{lem:universalCircIso} ist also auch $c_X \circ c_X^{-1} = 1_X \colon (X,\xi_A) \to (X,\xi)$ eine universelle Abbildung, denn da $(Y_A,\eta_A)$ und $(X,\xi_A)$ Elemente aus $|\A|$ sind, ist der $\C$-Morphismus $c_x$ insbesondere ein $\A$-Morphismus, da $\A$ eine volle Unterkategorie von $\C$ ist.

  Daher ist $((X,\xi_A), 1_X)$ die Coreflektion von $(X,\xi)$ bezüglich $\A$, man erhält also bis auf Isomorphie die Coreflektion eines  $\C$-Objekts $(X,\xi)$ bezüglich $\A$ durch eine Modifikation der $\C$-Struktur $\xi$ auf $X$.
\end{bem}

Wir schließen nun dieses Kapitel mit einem letzten Resultat zu allgemeinen Topologischen Konstrukten, welches eine Antwort auf die Frage liefert, wie sich initiale und finale Strukturen auf topologische Unterkonstrukte übertragen.

\begin{thm}[\cite{preuss}, 2.2.12]
  Sei $\A$ ein volles und unter Isomorphie abgeschlossenes Unterkonstrukt eines topologischen Konstrukts $\C$.
  Dann ist auch $\A$ topologisch, vorausgesetzt dass $\A$ bireflektiv oder bikoreflektiv in $\C$ ist.
  
  Ist $\A$ bireflektiv (bicoreflektiv) in $\C$, dann stimmen die initialen (finalen) Strukturen in $\A$ mit denen aus $\C$ überein, während die finalen (initialen) Strukturen in $\A$ aus den finalen (initialen) Strukturen in $\C$ entstehen, indem man den Bireflektor (Bicoreflektor) anwendet.
\end{thm}

