\section{Reflektive und coreflektive Unterkategorien}

In diesem Abschnitt wollen wir uns speziell mit Inklusionsfunktoren und ihren Adjungierten beschäftigen. 
In der freien Wildbahn treten Inklusionsfunktoren bei der Betrachtung von Unterkategorien auf.

\begin{defn}[Reflektive Unterkategorie]
  Sei $A$ eine Unterkategorie einer Kategorie $\C$ und $\F_e \colon \A \to \C$ der Inklusionsfunktor.
  Dann nennen wir $\A$ \emph{reflektiv} in $\C$ genau dann, wenn eine der folgenden äquivalenten Bedingungen erfüllt ist:
  \begin{enumerate}[(1)]
    \item $\F_e$ besitzt den Linksadjungierten Funktor $\R$.
    \item Jedes $X \in |C|$ besitzt eine universelle Abbildung $(r_X, X_\A)$ bezüglich $\F_e$.
  \end{enumerate}
  Den Funktor $\R$ nennen wir dann einen \emph{Reflektor}, die Morphismen $r_X \colon X \to X_\A$ nennen wir Reflektionen von $X$ bezüglich $\A$.

  Durch Dualisierung erhalten wir einen weiteren Begriff.
  Wir nennen $\A$ \emph{coreflektiv} in $\C$, genau dann, wenn $\A^*$ reflektiv ist in $\C^*$. 
\end{defn}

\begin{defn}
  In der Situation von Definition ? nennen wir $\A$ \emph{epireflektiv/ extremal epireflektiv/ bireflektiv in} $\C$, falls $\A$ reflektiv in $\C$ ist und der für alle $X \in |\C|$ existierende $\C$-morphismus $r_X \colon X \to X_\A$ ein Epimorphismus/ extremaler Epimorphismus / Bimorphismus ist.
  Die Morphismen $r_X$ nennen wir \emph{Epireflektionen/ extremale Epireflektionen/ Bireflektionen}.
\end{defn}

  \begin{itemize}
    \item Reflektive Unterkategorie
    \item Reflektor
    \item epireflektive, extremal epireflektiv, bireflektive Unterkategorie
    \item Reflektionen
  \end{itemize}

\begin{defn}
  Wir nennen ein Objekt $S$ eine Kategorie $\C$ \emph{Separator}, falls für alle paarweise verschiedenen Morphismen $f,g \colon A \to B$ mit gleichem Definitions und Wertebereich ein Morphismus $h \colon S \to A$ existiert mit der Eigenschaft, dass $f \circ h \neq g \circ h$.
\end{defn}

Wir halten fest, dass jedes Objekt $(X,\xi)$ eines topologischen Konstrukts $\C$ mit $X \neq \emptyset$ ein Separator ist.

\begin{bem}[2.2.11, S.65]
  Jedes coreflektive, volle und unter Isomorphie abgeschlossene Unterkonstrukt $\A$ eines topologischen Konstrukts $\C$ ist bicoreflektiv, falls $|\A|$ mindestens ein Element mit nicht leerer zugrunde liegender Menge enthält.

  In diesem Fall ist die zu $(X, \xi) \in \C$ gehörige Coreflektion $c_X \colon (Y_\A, \eta_\A) \to (X, \xi)$ bijektiv.
  Nach \cite[1.2.2.7]{preuss} existiert eine $\C$-Struktur $\xi_\A$ auf $X$, sodass $c_X \colon (Y_\A, \eta_\A) \to (X, \xi_\A)$ ein Isomorphismus ist.
  Da $\A$ nach Voraussetzung abgeschlossen unter Isomorphismen ist, gilt $(X, \xi_\A)\in \A$.
  Wir zeigen nun, dass $\xi_\A$ die gröbste aller $\C$-Strukturen $\xi'$ ist, für die einerseits $\xi' \leq \xi$ und andererseits $(X, \xi') \in \A$ gilt.

\end{bem}

\begin{thm}[2.2.12, S.66]
  
\end{thm}

